\documentclass{beamer}
\usepackage[spanish]{babel}
\usepackage[utf8]{inputenc}
\usepackage{graphicx}

\usecolortheme[RGB={122,59,122}]{structure}

\newtheorem{definicion}{Definición}
\newtheorem{ejemplo}{Ejemplo}

\title{Manejo básico de beamer}
\author{Lara Kristjansdottir, Javier Hernández Pérez}
\date{\today}
\usetheme{Madrid}
\begin{document}
\begin{frame}
\titlepage
\end{frame}
%++++++++++++++++++++++++++++++++++++++++++++++++++++++++++++++++++++++++++++++++++++
%++++++++++++++++++++++++++++++++++++++++++++++++++++++++++++++++++++++++++++++++++++
\begin{frame}
  \frametitle{Contenido}
  \tableofcontents
\end{frame} 
%++++++++++++++++++++++++++++++++++++++++++++++++++++++++++++++++++++++++++++++++++++
%++++++++++++++++++++++++++++++++++++++++++++++++++++++++++++++++++++++++++++++++++++
\section{Motivación y objetivos}
\begin{frame}
\frametitle{Motivación}

El metodo de los trapecios y un terreno.
\end{frame}
%++++++++++++++++++++++++++++++++++++++++++++++++++++++++++++++++++++++++++++++  
\begin{frame}
  \frametitle{Bibliografía}

  \begin{thebibliography}{10}

    \beamertemplatebookbibitems
    \bibitem[libro]{libro}  
    Juan de Burgos Román (2007) Cálculo infinitesimal de una variable segunda edición McGraw Hill
    
      
    
      \end{thebibliography}
\end{frame}


%++++++++++++++++++++++++++++++++++++++++++++++++++++++++++++++++++++++++++++++  

\end{document}