%%%%%%%%%%%%%%%%%%%%%%%%%%%%%%%%%%%%%%%%%%%%%%%%%%%%%%%%%%%%%%%%%%%%%%%%%%%%%%%
% Chapter 3: Procedimiento experimental 
%%%%%%%%%%%%%%%%%%%%%%%%%%%%%%%%%%%%%%%%%%%%%%%%%%%%%%%%%%%%%%%%%%%%%%%%%%%%%%%

%++++++++++++++++++++++++++++++++++++++++++++++++++++++++++++++++++++++++++++++
\section{Descripci�n de los experimentos}
\label{3:sec:1}

Vamos a aplicar la regla del trapecio a la funci�n sin(pi*x) en el intervalo [-2,-1] utilizando una cantidad variable de subintervalos, n. Para cada valor de n, mediremos el error absoluto y el tiempo de ejecuci�n del m�todo. Adicionalmente, aplicaremos la regla de Simpson a la misma funci�n con iguales par�metros para comparar ambos m�todos.

%++++++++++++++++++++++++++++++++++++++++++++++++++++++++++++++++++++++++++++++
\section{Descripci�n del material}
\label{3:sec:2}

Los experimentos han tenido lugar sobre un procesador Intel Core i3-2350M a 2.30 GHz. N�tese que aunque la m�quina tiene cuatro procesadores, s�lo se emple� uno para realizar los c�lculos. El sistema operativo fue Linux Mint 14.1 (Nadia) con la versi�n 3.5.0-17 del kernel. Los algoritmos fueron implementados en Python y la versi�n del int�rprete fue la 2.7.3.

%++++++++++++++++++++++++++++++++++++++++++++++++++++++++++++++++++++++++++++++
\section{Resultados obtenidos}
\label{3:sec:3}

Aqu� vamos a esperar hasta tener la tabla resumen, las gr�ficas y el c�lculo anal�tico de la cota del error.

%------------------------------------------------------------------------------
%\begin{figure}[!th]
%\begin{center}
%\includegraphics[width=0.75\textwidth]{images/figura1.eps}
%\caption{Ejemplo de figura}
%\label{fig:1}
%\end{center}
%\end{figure}
%------------------------------------------------------------------------------


%------------------------------------------------------------------------------
%\input{tables/table.tex}
%------------------------------------------------------------------------------

%++++++++++++++++++++++++++++++++++++++++++++++++++++++++++++++++++++++++++++++
\section{An�lisis de los resultados}
\label{3:sec:4}

bla, bla, etc. 

