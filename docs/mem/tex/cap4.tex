%%%%%%%%%%%%%%%%%%%%%%%%%%%%%%%%%%%%%%%%%%%%%%%%%%%%%%%%%%%%%%%%%%%%%%%%%%%%%
% Chapter 4: Conclusiones y Trabajos Futuros 
%%%%%%%%%%%%%%%%%%%%%%%%%%%%%%%%%%%%%%%%%%%%%%%%%%%%%%%%%%%%%%%%%%%%%%%%%%%%%%%
\section{Conclusiones}
\label{4:sec:1}
El m\'etodo del trapecio nos da una buena aproximaci\'on de la integral que quer\'iamos calcular con un coste temporal razonable. El error obtenido nunca supera la cota calculada y con m\'as subintervalos el error absoluto disminuye y la aproximaci\'on es m\'as precisa aunque se tarde m\'as.
