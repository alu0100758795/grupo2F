%%%%%%%%%%%%%%%%%%%%%%%%%%%%%%%%%%%%%%%%%%%%%%%%%%%%%%%%%%%%%%%%%%%%%%%%%%%%%
% Chapter 1: Motivaci�n y Objetivos 
%%%%%%%%%%%%%%%%%%%%%%%%%%%%%%%%%%%%%%%%%%%%%%%%%%%%%%%%%%%%%%%%%%%%%%%%%%%%%%%

%Los objetivos le dan al lector las razones por las que se realiz� el
%proyecto o trabajo de investigaci�n.
%---------------------------------------------------------------------------------
\section{Utilidad el m�todo}
\label{1:sec:1}
 % Primer p�rrafo de la primera secci�n.
No todas las integrales son f�ciles de resolver. Algunas son, de hecho, imposibles de calcular anal�ticamente. Por ello, es muy importante tener m�todos que nos permitan aproximar num�ricamente las integrales cuando tenemos una integral definida. En este caso nos centraremos en el m�todo de los trapecios.\par

Adem�s, este m�todo tiene una importancia extra. Si un m�todo, como este, es sencillo.Teniendo en cuenta que con funciones en la vida real, midiendo un terreno por ejemplo,normalmente (salvo casos particulares como por ejemplo una persona con un terreno cuadrado que quiera que cada uno de sus dos hijos hereden la mitad de su terreno dividido por la funci�n $sen(\pi x)$) estas haciendo una aproximaci�n.Usar directamente el m�todo de los trapecios parece una opci�n l�gica.\par

Por ejemplo hallas el �rea de un cuadrado al medio del terreno y vas eligiendo puntos de la frontera del terreno que esten unidos aproximadamente por lineas rectas. Ya solo queda hallar el area de los trapecios con base el lado del cuadrado que les quede m�s cerca y sumar.
%---------------------------------------------------------------------------------
\section{Nuestra integral $f(x)=sen(\pi x)$, $x \in [-2,-1]$ se puede resolver}
\label{1:sec:2}
  %Primer p�rrafo de la segunda secci�n.
  Nuestra integral es f�cil de resolver anal�ticamente, lo cual es una gran ventaja ya que nos permite comprobar c�mo de buenas ser�n nuestras aproximaciones. Para ello simplemente hemos de restar al valor calculado anal�ticamente el resultado de la aproximaci�n por el m�todo de los trapecios \par
  
  Resolvamos ahora $\int _{-2}^{-1}sen(\pi x)dx$. Haciendo el cambio $y=\pi x$ teniendo en cuenta que $\frac {\partial y} {\partial x}=\pi$ , $ x=-2 \Rightarrow y=-2 \cdot \pi$ , $ x=-1 \Rightarrow y=-\pi$ obtenemos que $\int _{-2}^{-1} sen(\pi x) dx=\frac{\int _{-2 \pi}^{-1\pi}sen(y)dy}{\pi}=\frac{-cos(-\pi)+cos(-2 \cdot \pi)}{\pi}=\frac{1+1}{\pi}=\frac{2}{\pi}\simeq 0,6366197724$
