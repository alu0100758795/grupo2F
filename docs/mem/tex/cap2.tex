%%%%%%%%%%%%%%%%%%%%%%%%%%%%%%%%%%%%%%%%%%%%%%%%%%%%%%%%%%%%%%%%%%%%%%%%%%%%%%%
% Chapter 2: Fundamentos Te�ricos 
%%%%%%%%%%%%%%%%%%%%%%%%%%%%%%%%%%%%%%%%%%%%%%%%%%%%%%%%%%%%%%%%%%%%%%%%%%%%%%%

%++++++++++++++++++++++++++++++++++++++++++++++++++++++++++++++++++++++++++++++

%En este cap�tulo se han de presentar los antecedentes te�ricos y pr�cticos que
%apoyan el tema objeto de la investigaci�n.

%++++++++++++++++++++++++++++++++++++++++++++++++++++++++++++++++++++++++++++++

\section{M�todo de los trapecios}
\label{2:sec:1}
El m�todo de los trapecios consiste en evaluar la funci�n a integrar en algunos puntos que se unen con l�neas rectas para aproximar el resto. Si unimos el eje y los puntos evaluados con rectas perpendiculares al eje. Hemos obtenido los trapecios formados por las rectas perpendiculares a los ejes, el eje y las rectas que unen a los puntos evaluados.La suma del area de esos trapecios es muy parecida a el area de debajo de la funci�n es decir la integral. El �rea de esos trapecios es facil de hallar$\frac {h} {2} (a+b)$ donde $h$ es la longitud del eje entre la imagen en el eje de los dos puntos por la recta perpendicular al punto, $a$ y $b$ la evaluaci�n de la funci�n en los dos puntos.

\section{Convergencia del m�todo de los trapecios}
\label{2:sec:2}
  %Primer p�rrafo de la segunda secci�n.
\textbf{Teorema:}\\
Sea $f$:$[a,b]$ $\rightarrow$ $\mathbb{R}$ una funci�n integrable en el intervalo cerrado $[a,b]$. Si $P=\{x_0,x_1,x_2,...,x_n\}$ es la partici�n de $[a,b]$ que divide a este intervalo en n partes iguales, entonces el siguiente n�mero aproxima a la integral $\int_a^bf(x)dx$.\\

$I_t= \frac{b-a}{n}\left( \frac{f(a)}{2} + f(x_1)+f(x_2)+f(x_3)+...+f(x_{n-1})+\frac{f(b)} {2} \right)$\\

La aproximaci�n tiene un error de si la funci�n $f$ es $C^2$ (derivable 2 veces con derivada continua).\\

$|I_t-\int_a^bf(x)dx|\leq \frac {(b-a)^3} {12n^2} f'' (\xi)$\\

Con $\forall \xi \in [a,b]$, luego si K es una cota superior de la derivada segunda en cualquier punto de $[a,b]$ tenemos:\\

$|I_t-\int_a^bf(x)dx|\leq \frac {(b-a)^3} {12n^2} K$ \\


Aunque esto no lo vamos a demostrar (pues es una cota mala y $f(x)=sen(\Pi x)$ es $C^\infty$ ) si $f$ no es $C^2$ solo pidiendole monoton�a en el intervalo se tiene:

$|I_t-\int_a^bf(x)dx|\leq \frac {b-a} {n} |f(b)-f(a)|$ \\

Demostraci�n:\\

Primero veamos que:
$I_t= \frac{b-a}{n}\left( \frac{f(a)}{2} + f(x_1)+f(x_2)+f(x_3)+...+f(x_{n-1})+\frac{f(b)} {2} \right)$\\
aproxima a la funci�n\\

Esto es cierto pues $\frac {b-a}{n}$ es la parte del eje x del trapecio entre cada 2 puntos.Si haces todos los trapecios cada punto se repite 2 veces salvo los extremos asi que son solo los extremos los que se dividen entre 2 cuando sumas todas las areas de los trapecios.\\


Ahora veamos que $|I_t-\int_a^bf(x)dx|\leq \frac {(b-a)^3} {12n^2} K$ es una cota del error\\

Sea $\alpha$ un punto de la partici�n P sea $h=\frac {b-a} {n}$ por simplificar notaci�n.\\

Cojamos una funci�n que nos lo hace en intervalos m�s peque�os que los h para ir hallando el error. \\

$F_\alpha :[0,h]\rightarrow \mathbb{R}$\\
$t\rightarrow F_\alpha(t)=\frac{t}{2} \left(f(\alpha)+f(\alpha+t) \right)-\int_\alpha ^{\alpha+t} f dt$\\

Derivando obtenemos:\\
$F'_\alpha(t)=\frac{1}{2}\left(f(\alpha)+f(\alpha+t) \right) +\frac{t}{2} 0 + \frac{t}{2} f'(\alpha)+t  -f(\alpha+t) +0=\frac{1}{2}\left(f(\alpha)+f(\alpha+t) \right)+ \frac{t}{2} f'(\alpha+t)  -f(\alpha+t)=\frac {1}{2} \left(f(\alpha)-f(\alpha+t) \right)+ \frac{t}{2} f'(\alpha+t)$\\

Derivando de nuevo \\


\section{Realizaci�n por el m�todo de los trapecios de $f(x)=sen(\Pi x)$, $x \in [-2,-1]$}
\label{3:sec:3}
$f(x)=sen(\Pi x)$, $x \in [-2,-1]$