%%%%%%%%%%%%%%%%%%%%%%%%%%%%%%%%%%%%%%%%%%%%%%%%%%%%%%%%%%%%%%%%%%%%%%%%%%%%%%%
% Chapter 2: Fundamentos Te�ricos 
%%%%%%%%%%%%%%%%%%%%%%%%%%%%%%%%%%%%%%%%%%%%%%%%%%%%%%%%%%%%%%%%%%%%%%%%%%%%%%%

%++++++++++++++++++++++++++++++++++++++++++++++++++++++++++++++++++++++++++++++

%En este cap�tulo se han de presentar los antecedentes te�ricos y pr�cticos que
%apoyan el tema objeto de la investigaci�n.

%++++++++++++++++++++++++++++++++++++++++++++++++++++++++++++++++++++++++++++++

\section{M�todo de los trapecios}
\label{2:sec:1}
El m�todo de los trapecios consiste en evaluar la funci�n a integrar en algunos puntos que se unen con l�neas rectas para aproximar el resto. Si unimos el eje y los puntos evaluados con rectas perpendiculares al eje. Hemos obtenido los trapecios formados por las rectas perpendiculares a los ejes, el eje y las rectas que unen a los puntos evaluados.La suma del area de esos trapecios es muy parecida a el area de debajo de la funci�n es decir la integral. El �rea de esos trapecios es facil de hallar$\frac {h} {2} (a+b)$ donde $h$ es la longitud del eje entre la imagen en el eje de los dos puntos por la recta perpendicular al punto, $a$ y $b$ la evaluaci�n de la funci�n en los dos puntos.

\section{Convergencia del m�todo de los trapecios}
\label{2:sec:2}
  %Primer p�rrafo de la segunda secci�n.
Teorema:\\

Sea f:

\section{Realizaci�n por el metodo de los trapecios de $f(x)=sen(\Pi x)$, $x \in [-2,-1]$}
\label{3:sec:3}
$f(x)=sen(\Pi x)$, $x \in [-2,-1]$